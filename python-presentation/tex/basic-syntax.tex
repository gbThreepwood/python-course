
\begin{frame}[containsverbatim]{Grunnleggande syntaks}

  Python bruker innrykk for å gruppera kode. Som til dømes i denne funksjonen:

\begin{minted}{python}
def min_funksjon():
  print("Melding frå funksjonen min_funksjon")
  svar = 2+2
  print("To pluss to er " str(svar))

min_funksjon()
\end{minted}
  
\end{frame}

\begin{frame}{Variablar og datatypar}

  I Python treng du normalt ikkje spesifisera kva datatype du ynskjer når du oppretter ein variabel. Dette er implisitt i tilordninga av verdi til variabelen (på samme måte som i Matlab). Blandt dei innebygde datatypane finn ein:
  
  \begin{itemize}
  \item int
  \item float
  \item str
  \item list
  \item tuple
  \item set
  \item dict
  \item book
  \item bytearray
    \item +mange fleire
  \end{itemize}
  
\end{frame}

\begin{frame}[containsverbatim]{Valgsetningar}

  Dette er sikkert sjølvforklarande for alle som har programmert i eit anna språk tidlegare:
  
\begin{minted}{python}
a = 50
b = 45
if b > a:
  print("b er størst")
elif a == b:
  print("a og b er like")
else:
  print("a er størst")
\end{minted}
  
\end{frame}

\begin{frame}[containsverbatim]{Løkker}

For-løkker:

\begin{minted}{python}
for i in range(0, 5):
	print(i)
\end{minted}

Iterasjon over element i lister:

\begin{minted}{python}
a = [3, 4, 5, 6, 8, 9, 20, 34]

for x in a:
	print(x+2)
\end{minted}
  
\end{frame}

\begin{frame}[containsverbatim]{Løkker}
	While-løkker:
	
\begin{minted}{python}
i = 0
while (i < 4):
	i = i + 1
	print("Talet er " + str(i))
else:
	print("Ferdig")
\end{minted}
	
\end{frame}

\begin{frame}[containsverbatim]{Komplekse tal}

  Komplekse tal i Python er internt lagra på rektangulær form, til dømes \(5 + 7j\). Python nyttar \(j = \sqrt{-1}\) i staden for \(i\).

\begin{minted}{python}
  z = 3 + 4j
  u = 230
  i = u/z
  import cmath
  cmath.polar(i)
\end{minted}
  
\end{frame}
%%% Local Variables:
%%% mode: latex
%%% TeX-master: "../python-presentation"
%%% End:
