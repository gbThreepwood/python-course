\begin{frame}{Motivasjon}
  Eg har planlagt ein presentasjon med fokus på Python som eit alternativ til Matlab. Studentane våre treng eit verktøy for å gjera utrekningar, og kompilerte språk som C/C++/C\# er etter mitt syn \textit{ikkje} egna til det.

  Dersom målet er å læra studentane opp i korleis datamaskiner fungerer, så er eg einig med dei som seier at Python er eit dårleg val. Men eg trur Python kan dekka dei fleste behov som Matlab dekker for oss i dag, og at det er ein god del ein kan få til i Python som ikkje er mulig (eller i det minste ikkje er enkelt) i Matlab.

  Nokon veldig sterke argument for Python handlar om tilgjengelighet:
  \begin{itemize}
    \item Dersom du manglar Python (og maskina di ikkje er låst inne i eit HVL domene) så kan du installera det på berre nokon få minutt.
    \item Det er kostnadsfritt, så studentane kan bruka det etter dei er uteksaminert.
    \item Du kan bruka det uten å ha tilgong til internett.    
  \end{itemize}

  
\end{frame}
%%% Local Variables:
%%% mode: latex
%%% TeX-master: "../python-presentation"
%%% End:
