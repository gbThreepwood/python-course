\begin{frame}{Installasjon}

  Det finnes svært mange måtar å installea python på. For å gjera det enkelt kan dei som nyttar Windows installera Anaconda: \href{https://www.anaconda.com/download/}{anaconda.com/download}.

  Dersom du køyrer GNU/Linux så er Python mest sannsyleg allereie installert. Det kan derimot likevel vera aktuelt å installera fleire utgåver av Python (til dømes både versjon 2 og 3). Dette er tatt høgde for, og er enkelt å få til.
\end{frame}

\begin{frame}{Interaktiv Python tolkar}

  Dersom du køyrer kommandoen \mintinline{console}{python} i ein terminal, så vil du bli presentert med ein interaktiv Python-tolkar. Dette kan vera nyttig får å testa syntaksen til ein funksjon eller anna kode du jobbar med, og det er også ein veldig god kalkulator dersom du har behov for å gjera nokon kjappe utrekningar.

Dersom det er installert, så vil \mintinline{console}{ipython} gi deg ein interaktiv python tolkar med litt meir fancy output. Til dømes syntaks fargar og ``autocomplete''.
  
\end{frame}

\begin{frame}{Teksteditor for koding i Python}

  Dersom du skal skriva lengre kode så har du svært mange program å velga mellom. Til dømes:
  
  \begin{itemize}
  \item Spyder
  \item VSCode
  \item PyCharm
  \item Jupyter
  \item Vim eller Emacs
  \end{itemize}
\end{frame}

%%% Local Variables:
%%% mode: latex
%%% TeX-master: "../python-presentation"
%%% End:
