\documentclass[10pt,a4paper]{beamer}
%\usepackage[utf8]{inputenc}
\usepackage[T1]{fontenc}
\usepackage[nynorsk]{babel}
\usepackage{amsmath}
\usepackage{amssymb}
\usepackage{graphicx}


\usepackage[
backend=biber,
style=ieee,
sorting=ynt
]{biblatex}

%\addbibresource{bib/latex-kurs-referansar.bib}

\usepackage{svg}

\usepackage{subfiles}

\usepackage{multicol}

\usetheme{metropolis}

\setbeamertemplate{section in toc}[sections numbered]
\setbeamertemplate{subsection in toc}[subsections numbered]

\usepackage[pagewise]{lineno}
\usepackage{csquotes}

\usepackage{minted}
\usepackage{listings}

\usepackage{multirow}

\usepackage{siunitx}
\usepackage{steinmetz}

\usepackage{xcolor}

\usepackage{enumerate}

\usepackage{lipsum}

\usepackage{caption}
\usepackage{subcaption}

\usepackage{pgfplots}


\title{Python for elektrofag}
\date{\today}
\author{Eirik Haustveit}
\institute{Institutt for datateknologi, elektroteknologi og realfag}

\begin{document}
	
	\titlepage

	\section{Plan}

	\begin{frame}{Plan}
		
          \begin{itemize}
                        \item Motivasjon
			\item Historikk og litt oversikt
                        \item Installasjon av Python
                        \item Grunnleggande syntaks
                        \item Komplekse tal
                        \item Numpy, sympy, scipy
			\item Demonstrasjon av nokre elektro pakkar
		\end{itemize}
		
	\end{frame}


	\subfile{tex/motivation.tex}
	
	\subfile{tex/history.tex}
	
	\subfile{tex/introduction.tex}	

    \subfile{tex/basic-syntax.tex}

    \subfile{tex/numpy.tex}

    \subfile{tex/sympy.tex}

    \subfile{tex/scipy.tex}

    \subfile{tex/matplotlib.tex}
    
  %  \subfile{tex/pandas.tex}
    
  %  \subfile{tex/lcapy.tex}

%	\begin{frame}{Referansar}	
	
%	\printbibliography 
	
%	\end{frame}
	
	
\end{document}
%%% Local Variables:
%%% TeX-command-extra-options: "--shell-escape --syntex=1"
%%% mode: latex
%%% TeX-master: t
%%% End:
