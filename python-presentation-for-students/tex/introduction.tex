\begin{frame}{Introduksjon}
	
	
	Python vert beskrive som eit høgnivå språk for generelle formål. Det er eit såkalla dynamisk skrevet språk med ``garbage collection'', som betyr at eit programs gyldighet vert sjekka etter kvart som det køyrer, og at minne som ikkje lenger er i bruk automatisk vert frigjort.
	
	Språket støttar i større eller mindre grad fleire ulike programmeringsparadigmer, prosedyrisk programmering, objektorientert programmering, og funksjonell programmering.
	
	Ein beskriv ofte python som eit språk med ``batteries included'', sidan det fylgjer med eit ganske omfattande standardbibliotek med funksjonar for både vanlege og meir obskure oppgåver.
	
\end{frame}


\begin{frame}{Installasjon}

  Det finnes svært mange måtar å installea Python på. Men dette skal forhåpentlegvis dei fleste av dykk allereie ha blitt ferdige med på tysdag.
  
  \begin{itemize}
  	\item Python kan installerast direkte
  	\item Det kan installerast via ein pakkebehandlar (som til dømes Anaconda)
  	\item Det kan vera aktuelt å installera fleire utgåver av Python (til dømes både versjon 2 og 3). Dette er tatt høgde for i oppbygginga av installasjonssystemet, og er enkelt å få til.
  	\item Dersom du køyrer GNU/Linux så er Python mest sannsyleg allereie installert.
  \end{itemize}
  
  %For å gjera det enkelt kan dei som nyttar Windows installera Anaconda: \href{https://www.anaconda.com/download/}{anaconda.com/download}.

  
\end{frame}

\begin{frame}{Interaktiv Python tolkar}

  Dersom du køyrer kommandoen \mintinline{console}{python} i ein terminal, så vil du bli presentert med ein interaktiv Python-tolkar. Dette kan vera nyttig får å testa syntaksen til ein funksjon eller anna kode du jobbar med, og det er også ein veldig god kalkulator dersom du har behov for å gjera nokon kjappe utrekningar.

Dersom det er installert, så vil \mintinline{console}{ipython} gi deg ein interaktiv python tolkar med litt meir fancy output. Til dømes syntaks fargar og ``autocomplete''.
  
\end{frame}

\begin{frame}{Teksteditor for koding i Python}

  Eit Python-program (Python kode) består av tekst. Dersom du skal skriva lengre kode så bør du bruka eit tekstredigeringsprogram. Her har du svært mange program å velga mellom. Til dømes:
  
  \begin{itemize}
  \item Spyder
  \item VSCode
  \item PyCharm
  \item Jupyter
  \item Vim eller Emacs
  \end{itemize}
  
  Du kan til og med bruka Microsoft Word. Det vil eg ikkje anbefala, men poenget er at så lenge programmet kan redigera ein tekstfil så vil det fungera for skriving av Python kode.
\end{frame}


\begin{frame}{Installasjon av pakkar til Python}
	
	Kommandoen \mintinline{console}{pip} kan nyttast til å installera ekstra pakkar til python. Til dømes \mintinline{console}{pip install ipython} for IPython. Nettsida \url{https://pypi.org/} gir deg tilgong til ein pakkebrønn med svært mange Python pakkar.
	
\end{frame}

%%% Local Variables:
%%% mode: latex
%%% TeX-master: "../python-presentation"
%%% End:
