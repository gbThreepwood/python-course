\begin{frame}{Motivasjon}

  Tidlegare har Matlab vore nytta som programmeringsspråk/kalkulator for dei fleste elektrofaga (og for mange av dei andre ingeniørretningane ved HVL), men frå og med i haust er det beslutta å gå over til Python.

  Python er meir enn eit alternativ til Matlab, men i denne presentasjonen er fokus korleis ein kan nytta Python som regneverktøy for ingeniørar innan elektrofag.

  Nokon veldig sterke argument for Python handlar om tilgjengelighet:
  \begin{itemize}
    \item Dersom du manglar Python (og maskina di ikkje er låst inne i eit HVL domene) så kan du installera det på berre nokon få minutt.
    \item Det er kostnadsfritt, så du kan bruka det etter du er uteksaminert.
    \item Du kan bruka det uten å ha tilgong til internett.
  \end{itemize}

  
\end{frame}
%%% Local Variables:
%%% mode: latex
%%% TeX-master: "../python-presentation"
%%% End:
